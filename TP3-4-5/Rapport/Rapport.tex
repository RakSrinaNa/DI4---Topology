\documentclass{report}
\usepackage{MCC}

\def\footauthor{Thomas COUCHOUD \& Victor COLEAU}
\title{Rapport TP3-4-5}
\author{Thomas COUCHOUD\\\texttt{thomas.couchoud@etu.univ-tours.fr}\\Victor COLEAU\\\texttt{victor.coleau@etu.univ-tours.fr}}

\begin{document}
	\mccTitle[no]
	\tableofcontents

	\chapter{OpenSM}
		Lors du premier lancement d'opensm, on nous indique que le module ib\_umad n'est chargé.
		
		Nous avons donc edité le fichier \cbo{/etc/modules} en y ajoutant la ligne "ib\_umad" puis redémarrant.
		
		Le \textbf{LID} correspond à Local Identifier et permet d'identifier chaque périphérique du réseau et l'utiliser comme adresse.
		
		Lors du lancement d'OpenSM, on remarque que les LIDs sont attribués.
	
	\chapter{Tables de routage}
		Un fichier route se décompose en les différentes tables de routages pour chaque commutateur.
		
		Chaque table contient des lignes similaires qui indique une route. Voici un exemple:
		
		0x0001 004 : (Channel Adapter portguid 0x0000000000100001: 'HCA(000)')
		
		Les champs sont les suivants:
		\begin{easylist}
			& Le LID de la destination
			& Le port de sortie sur le switch
			& Des informations sur la destination, notamment son nom
		\end{easylist}

	
	\chapter{Etudes de routages}
		\section{Etudes de topologies avancés}
			Le fichier \cbo{Topo1.topo} contient une topologie $PGFT\left( 2,\left[ 3,3\right], \left[ 1,2\right], \left[ 1,2\right]\right)$.
			
			Le fichier \cbo{Topo2.topo} contient une topologie $XGFT\left( 3, \left[ 1,2,2\right], \left[ 4,2,2\right]\right)$.
			
			\begin{table}[H]%TODO
				\begin{tabularx}{0.99\textwidth}{|c|X|X|}
					\hline
					& XGFT & PGFT\\\hline\endhead
					Avantages & On peut multiplier les liens entre des n\oe ux du niveau $L$ et $L-1$. Non bloquant. Premier "petaflopique".& \\\hline
					Inconvénients & On ne peut pas doubler les liens entre deux n\oe ux & \\\hline	
				\end{tabularx}
				\caption{Avantages et inconvénients des topologies précédentes}
			\end{table}
			
			Les tables de routage générées sont disponibles dans \autoref{sec:routes}.
						
		\section{Quelques algorithmes de routage}
		
		\section{Quelques métriques pour l’étude des performances des algorithmes de routage}
			\section{MinHop}
				Le code du calculateur est disponible dans \autoref{sec:code}
				
			\section{Route balancing}
				
			
		
	\appendix
	\chapter{Calculateur\label{sec:code}}
		\lstinputlisting[caption="Calculation.cpp"]{../Outils/calculation.cpp}
	
	\chapter{Tables de routage\label{sec:routes}}
		\section{Topologie 1}
			\lstinputlisting[caption="Route 1"]{../route1.route}
			
		\section{Topologie 2}
			\lstinputlisting[caption="Route 2"]{../route2.route}
	
\end{document}