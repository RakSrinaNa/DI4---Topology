\documentclass{report}
\usepackage{MCC}

\def\footauthor{Thomas COUCHOUD \& Victor COLEAU}
\title{Rapport TP1}
\author{Thomas COUCHOUD\\\texttt{thomas.couchoud@etu.univ-tours.fr}\\Victor COLEAU\\\texttt{victor.coleau@etu.univ-tours.fr}}

\begin{document}
	\mccTitle[no]
	\tableofcontents

	\chapter{Ibsim}
		Après avoir recopié le fichier de configuration topologique et lancé le programme avec ce dernier, on peut observer que la sortie dans la console correspond exactement avec la figure 2 du sujet.
		
		\img[fig:dump]{dump.png}{Sortie du programme avec dump}{0.7}
		
		On peut par la suite rentrer de nouvelles commandes. La liste de ces dernières est disponible en tapant \cbo{help}.
		
		\begin{easylist}[itemize]
			& !: Lance des commandes depuis un fichier
			& dump: Affiche les informations du réseau
			& Route: Affiche la route entre deux noeuds
			& Link: permet de faire une connexion réseau.	
			& Relink: Réétabli une connexion réseau.
			& Unlink: Supprime un lien réseau.
			& Clear: Supprime les liens réseau \& réinitialise les ports.
			& Guid: Définit le GUID d'un noeud.
			& Error: Définit le statut d'erreur d'un port/noeud.
			& PerformanceSet: Définit les conteurs de performance.
			& Baselid: Change le lid d'un port.
			& Verbose: Définit le niveau de sortie dans la console.
			& Wait: Suspend la simulation.
			& Attached: Liste tous les clients attachés.
			& X: Déconnecte un client.
			& Quit: Ferme le programme.
		\end{easylist}
		
		Nous avons tenté route mais on nous dit que les lid ne sont pas définis alors que nous les avons définis pour chaque n\oe ud.
	
	\chapter{Générateur de topologie}
		Tout d'abord le dossier de sources contient un Makefile. Il est donc possible de compiler facilement le programme grâce à la commande \cbo{make}.
		
		Plusieurs classes sont présentes:
		\begin{easylist}[itemize]
			& Node: représente un n\oe ud du réseau. Ce dernier est composé d'un nom, type, ID, et d'une liste de ports avec leurs connections.
			& Machine: représente une machine du réseau. Cette classe étend Node avec comme type "Hca" et 1 port.
			& Commutator: représente un switch du réseau. Cette classe étend Node avec comme type "Switch".
		\end{easylist}
	
		Le fichier main réalise la création des éléments ainsi que les liens entre ces derniers:
		\begin{easylist}[itemize]
			& Création de toutes les machines
			& Création de tout les switchs
			&& Aggregation
			&&& Création de tout les switch dans la zone d'aggregation
			&&& Création des liens entre le switch et les machines qui le concernent
			&& Edge
			&&& Création de tout les switch dans la zone edge
			&&& Création des liens entre le switch et les switchs de la zone d'aggregation
			&& Core
			&&& Création de tout les switch dans la zone core
			&&& Création des liens entre le switch et les switchs de la zone edge
			& Ajustements des ports pour suivre la norme imposée par le PDF
		\end{easylist}
	
		Voici un exemple de fichier de sortie pour $k=4$:
		\lstinputlisting[caption=topo.topo]{../Generator/topo.topo}

	\appendix
	\chapter{Sources du générateur}
		\section{Node}
			\lstinputlisting[caption=Node.h]{../Generator/Node.h}
			\lstinputlisting[caption=Node.cpp]{../Generator/Node.cpp}
			
		\section{Machine}
			\lstinputlisting[caption=Machine.h]{../Generator/Machine.h}
			\lstinputlisting[caption=Machine.cpp]{../Generator/Machine.cpp}
			
		\section{Commutator}
			\lstinputlisting[caption=Commutator.h]{../Generator/Commutator.h}
			\lstinputlisting[caption=Commutator.cpp]{../Generator/Commutator.cpp}
			
		\section{Main}
			\lstinputlisting[caption=main.cpp]{../Generator/main.cpp}
		
\end{document}