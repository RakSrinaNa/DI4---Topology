\documentclass{report}
\usepackage{MCC}

\def\footauthor{Thomas COUCHOUD \& Victor COLEAU}
\title{Rapport TP3-4-5}
\author{Thomas COUCHOUD\\\texttt{thomas.couchoud@etu.univ-tours.fr}\\Victor COLEAU\\\texttt{victor.coleau@etu.univ-tours.fr}}

\begin{document}
	\mccTitle[no]
	\tableofcontents

	\chapter{OpenSM}
		Lors du premier lancement d'opensm, on nous indique que le module ib\_umad n'est pas chargé.
		
		Nous avons donc edité le fichier \cbo{/etc/modules} en y ajoutant la ligne "ib\_umad" puis redémarrer.
		
		Le \textbf{LID} correspond à Local Identifier et permet d'identifier chaque périphérique du réseau et de l'utiliser comme adresse.
		
		Lors du lancement d'OpenSM, on remarque que les LIDs sont attribués.
	
	\chapter{Tables de routage}
		Un fichier route se décompose en les différentes tables de routages pour chaque commutateur.
		
		Chaque table contient des lignes similaires qui indiquent une route. Voici un exemple:
		
		0x0001 004 : (Channel Adapter portguid 0x0000000000100001: 'HCA(000)')
		
		Les champs sont les suivants:
		\begin{easylist}
			& Le LID de la destination
			& Le port de sortie sur le switch
			& Des informations sur la destination, notamment son nom
		\end{easylist}

	
	\chapter{Etudes de routages}
		\section{Etudes de topologies avancés}
			Le fichier \cbo{Topo1.topo} contient une topologie $PGFT\left( 2,\left[ 3,3\right], \left[ 1,2\right], \left[ 1,2\right]\right)$.
			
			Le fichier \cbo{Topo2.topo} contient une topologie $XGFT\left( 3, \left[ 1,2,2\right], \left[ 4,2,2\right]\right)$.
			
			\begin{table}[H]%TODO
				\begin{tabularx}{0.99\textwidth}{|c|X|X|}
					\hline
					& XGFT & PGFT\\\hline\endhead
					Avantages & On peut multiplier les liens entre des n\oe ux du niveau $L$ et $L-1$. Non bloquant. Premier "petaflopique".& Permet d'éviter les phénomènes de congestion. Permet de réduire le coût par élagage des liens. Permet d'assurer une bonne bande passante.\\\hline
					Inconvénients & On ne peut pas doubler les liens entre deux n\oe ux & ---\\\hline	
				\end{tabularx}
				\caption{Avantages et inconvénients des topologies précédentes}
			\end{table}
			
			Les tables de routage générées sont disponibles dans \autoref{sec:routes}.
			
			\begin{figure}[H]
				\begin{minipage}{0.74\textwidth}
					\img{T1.png}{Route topologie 1}{0.65}
				\end{minipage}
				\begin{minipage}{0.25\textwidth}
					\begin{easylist}
						& Route 1 (rouge): La route prise démarre de 000 pour aller vers le seul switch auquel il est connecté. Puis ce switch étant connecté à la destination 010, le paquet prend la route <100> $\rightarrow$ 010.
						
						& Route 2 (vert): La route prise démarre de 000 pour aller vers le seul switch auquel il est connecté. Puis ce switch regarde sa table de routage et détermine qu'il faut l'envoyer à <001> par le port 5. Ce nouveau switch regarde sa table de routage et détermine qu'il faut l'envoyer à <110> par le port 5. Puis ce switch étant connecté à la destination 200, le paquet prend la route <110> $\rightarrow$ 200.
					\end{easylist}
				\end{minipage}
			\end{figure}
			
			\img{T2.png}{Route topologie 2}{0.5}
			
			\begin{easylist}
						& Route 1 (rouge): La route prise démarre de 000 pour aller vers le seul switch auquel il est connecté. Puis ce switch regarde sa table de routage et détermine qu'il faut l'envoyer à <100> par le port 3. Ce nouveau switch regarde sa table de routage et détermine qu'il faut l'envoyer à <201> par le port 2. Puis ce switch étant connecté à la destination 010, le paquet prend la route <201> $\rightarrow$ 010.
						
						& Route 2 (vert): La route prise démarre de 000 pour aller vers le seul switch auquel il est connecté. Puis ce switch regarde sa table de routage et détermine qu'il faut l'envoyer à <100> par le port 3. Puis ce switch regarde sa table de routage et détermine qu'il faut l'envoyer à <000> par le port 3. Puis ce switch regarde sa table de routage et détermine qu'il faut l'envoyer à <120> par le port 3. Puis ce switch regarde sa table de routage et détermine qu'il faut l'envoyer à <220> par le port 1. Puis ce switch étant connecté à la destination 200, le paquet prend la route <220> $\rightarrow$ 200.
					\end{easylist}
						
		\section{Quelques algorithmes de routage}
			\subsection{MinHop}
				Cet algorithme a pour but d'emprunter le plus court chemin en termes de sauts. Il cherche donc à optimiser le nombre de sauts.
				
			\subsection{UPDN}
				UPDN signifie "Up/Down" et fonctionne sur le même principe que minhop: emprunter la plus courte distance. Cependant une règle supplémentaire permet d'éviter notamment les bouclages: l'algorithme passe d'abord par une phase montante, puis une descendante. De cette manière il est interdit de redescendre dans l'arbre puis de remonter.
				
			\subsection{FTree}
				FTree reprend les principes d'UPDN tout en évitant les phénomènes de congestion.
				
			\subsection{DOR}
				Reprend la technique vue en cours avec le XOR dans un hypercube et permet ainsi de diriger le traffic dimension par dimension.
			
			\subsection{LASH}
				Mix entre UDPN et DOR. Permet de prendre le plus court chemin tout en répartissant le traffic entre les couches afin d'éviter le phénomène de congestion.
		
		\section{Quelques métriques pour l’étude des performances des algorithmes de routage}
			\section{MinHop}
				Le code du calculateur est disponible dans \autoref{sec:code}
				
			\section{Route balancing}
				Le code du calculateur est disponible dans \autoref{sec:code}
				
			\section{Résultats}
				Topologie 1: MinHop = 4, Balancing  4.
				
				Topologie 2: MinHop = 6, Balancing = 20.
				
				Après analyse de ces résultats, nous déduisons que sur la topologie n°2 il faudrait utiliser un algorithme de routage évitant les phénomènes de congestion. Nous préconisons l'utilisation de FTree ou LASH.
				
				En revanche, pour la topologie n°1, UPDN ou DOR semblent suffisants.
	\chapter{NS3}
		Par manque de temps (la compilation étant très longue, plus d'une heure avec les tests etc.) nous n'avons pas pu tester NS3.
		
		\img{NS3.png}{Tests NS3}{0.45}
		
	\appendix
	\chapter{Calculateur\label{sec:code}}
		\lstinputlisting[caption="Calculation.cpp"]{../Outils/calculation.cpp}
	
	\chapter{Tables de routage\label{sec:routes}}
		\section{Topologie 1}
			\lstinputlisting[caption="Route 1"]{../route1.route}
			
		\section{Topologie 2}
			\lstinputlisting[caption="Route 2"]{../route2.route}
	
\end{document}