\documentclass{report}
\usepackage{MCC}

\def\footauthor{Thomas COUCHOUD \& Victor COLEAU}
\title{Rapport TP3-4-5}
\author{Thomas COUCHOUD\\\texttt{thomas.couchoud@etu.univ-tours.fr}\\Victor COLEAU\\\texttt{victor.coleau@etu.univ-tours.fr}}

\begin{document}
	\mccTitle[no]
	\tableofcontents

	\chapter{OpenSM}
		Lors du premier lancement d'opensm, on nous indique que le module ib\_umad n'est chargé.
		
		Nous avons donc edité le fichier \cbo{/etc/modules} en y ajoutant la ligne "ib\_umad" puis redémarrant.
		
		Lors du second essaie, le message \cbo{No local ports detected}.
	
	\chapter{Tables de routage}
	
	\chapter{Etudes de routages}
		\section{Etudes de topologies avancés}
			Le fichier \cbo{topo1.topo} contient une topologie $PGFT\left( 2,\left[ 3,3\right], \left[ 1,2\right], \left[ 1,2\right]\right)$.
			
			Le fichier \cbo{topo2.topo} contient une topologie $XGFT\left( 3, \left[ 1,2,2\right], \left[ 4,2,2\right]\right)$.
			
			\begin{table}[H]%TODO
				\begin{tabularx}{0.99\textwidth}{|c|X|X|}
					\hline
					& XGFT & PGFT\\\hline\endhead
					Avantages & On peut multiplier les liens entre des n\oe ux du niveau $L$ et $L-1$. Non bloquant. Premier "petaflopique".& \\\hline
					Inconvénients & On ne peut pas doubler les liens entre deux n\oe ux & \\\hline	
				\end{tabularx}
				\caption{Avantages et inconvénients des topologies précédentes}
			\end{table}
			
		
\end{document}